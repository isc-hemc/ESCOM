\section{Problemas y Resolución}

Debo añadir que me causo un poco de dificultad instalar y compilar los programas extensión \textit{.l} y \textit{.y}, puesto que primero tuve que instalar \textit{flex} con Homebrew, independientemente de esto en general la práctica me resulto bastante sencilla puesto que tome como punto de partida el ejemplo visto en clase.

\section{Conclusiones}

Es muy interesante ver un programa que genere un analizador léxico y sintáctico que ademas valide las entradas y resuelva las operaciones señaladas. Sin duda una practica bastante interesante puesto que no esperaba que el ejecutable fuera un escáner esperando por una entrada que concuerde con cualquiera de sus patrones.

\pagebreak