\section{Desarrollo}

La practica consta de desarrollar un programa que sea capaz de generar autómatas finitos no deterministas por medio de una quintupla definida como:

\begin{equation}
(Q, \Sigma, \alpha, q_{0}, F)
\end{equation}

Donde:

\begin{itemize}
\item Q: Conjunto de estados finito y no vació.
\item $\Sigma$: Alfabeto de entrada.
\item $\alpha$: Función de transición.
\item $q_{0}$: Estado inicial.
\item F: Conjunto de estados finales. 
\end{itemize}

Estas quintuplas serán almacenadas en archivos TXT de tal forma que nuestro programa pueda leer el archivo y extraer los datos de el.

\subsection{Lectura del archivo}

El siguiente fragmento de código nos permite leer el archivo TXT, limpiar los datos en caso de que existan símbolos como saltos de linea o espacios en blanco y construir la quintupla que nos permitira generar el autómata finito que será capaz de aceptar cadenas pertenecientes a su alfabeto y decidir si tienen un estado de aceptación:

\begin{lstlisting}
class FileHandler:
    """File Handler.

    This class handles everything that has to do with the file where the
    automaton quintuple is located; the purpose of this class it's to read the
    automaton quintuple file, clean it from any `special character` stored in
    the file (like the backslash for example) and build the quintuple in a
    comprehensive tuple for the program `(states, alphabet, start_state,
    accept_states, transitions)`.

    Attributes
    ----------
    path: str
        Path where the quintuple file is located.

    """

    __slots__ = ("path",)

    def __init__(self, path: str):
        """Constructor."""
        self.path: str = path

    def read_file(self) -> Tuple:
        """Read file.

        Reads the file in the `path` attribute.

        Returns
        -------
        Tuple
            An AFN quintuple.

        """
        with open(self.path, "r") as f:
            data: List = f.readlines()
        data = self.clean_data(data)
        return self.build_quintuple(data)

    def build_quintuple(self, data: List) -> Tuple:
        """Build Quintuple.

        From the data collected from the file in the `path` attribute builds
        the automata quintuple.

        Parameters
        ----------
        data: List
            Raw data readed for the file in the `path` attribute.

        Returns
        -------
        Tuple
            Automata quintuple.

        """
        states: List = data.pop(0).split(",")
        alphabet: List = data.pop(0).split(",")
        start_state: str = data.pop(0)
        accept_states: List = data.pop(0).split(",")
        transitions: List = [element.split(",") for element in data]
        return (states, alphabet, start_state, accept_states, transitions)

    def clean_data(self, data: List) -> List:
        """Clean data.

        Removes the backslash character for each element into a list.

        Parameters
        ----------
        data: List
            Raw data readed for the file in the `path` attribute.

        Returns
        -------
        List

        """
        return [element.rstrip() for element in data]
\end{lstlisting}

\subsection{Algoritmo}

El siguiente algoritmo recursivo es el que permitirá generar el autómata finito y determinar si la cadena entrante pertenece o no al alfabeto. La clase AFN recibe como atributos la quintupla extraída del archivo TXT con el fragmento de código mostrado anteriormente, estos datos serán utilizados por la función \textbf{validate} que recibe como parámetros la cadena a validar, un apuntador a la posición actual de la cadena, el conjunto de caminos que se han generado y el estado actual en el que se encuentra el símbolo de la cadena que esta siendo evaluado.

\begin{lstlisting}
class AFN:
    """AFN class.

    Object that contains the information of an AFN.

    Attributes
    ----------
    states: List
        Finite set of states.
    alphabet: List
        Finite set of symbols.
    start_state: str
        State before any input been processed.
    accept_state: List
        Set of states.
    transitions: List
        Transition function

    """

    __slots__ = (
        "states",
        "alphabet",
        "start_state",
        "accept_states",
        "transitions",
        "_result",
        "_error",
    )

    def __init__(
        self,
        states: List,
        alphabet: List,
        start_state: str,
        accept_states: List,
        transitions: List,
    ):
        """Constructor."""
        self.states: List = states
        self.alphabet: List = alphabet
        self.start_state: str = start_state
        self.accept_states: List = accept_states
        self.transitions: List = transitions
        self._result: List = []
        self._error: str = "x"

    def _filter_transitions(
        self, data: str, position: int, current: str
    ) -> List:
        """Obtain Transition Set.

        Gets the possible next transitions that the automaton can follow for
        the given input.

        Parameters
        ----------
        data: str
            Input to validate.
        position: int
            Pointer to the current position of the input.
        current: str
            Current state of the automaton in which the input is positioned.

        Returns
        -------
        List
            Transitions that the automaton can follow for the input in the
            current position.

        """
        return list(
            filter(
                lambda x: x[0] == current and data[position] == x[1],
                self.transitions,
            )
        )

    def validate(
        self,
        data: str,
        position: int = 0,
        path: List = [],
        current: Optional[str] = None,
    ) -> List:
        """Validate.

        Validates if a given input belongs to the automaton alphabet.

        Parameters
        ----------
        data: str
            Input to validate.
        position: int
            Pointer to the current position of the input.
        path: List
            Route that the automaton is generating for the input.
        current: optional, str
            Current state of the automaton in which the input is positioned.

        Returns
        -------
        List
            All posible routes that the automaton generate for the input.

        """
        current = self.start_state if current is None else current
        if len(data) < position + 1:
            path.append(current)
            self._result.append(path)
            return path
        transition_set = self._filter_transitions(data, position, current)
        if not transition_set:
            path.append(current)
            self.validate(data, position + 1, path.copy(), self._error)
        path.append(current)
        for t in transition_set:
            self.validate(data, position + 1, path.copy(), t[2])
        return self._result
\end{lstlisting}
