\section{Problemas y Resolución}

Debo añadir que me causo bastante dificultad encontrar una solución a esta practica puesto que algo que no tenia contemplado era el estado de \textbf{ERROR} de un autómata, al tener esto en cuenta y generar la quintupla completa la validación fue mucho mas sencilla, como consecuencia, mi programa genera un archivo llamado \textit{n-extended.txt} el cual es la misma quintupla pero con su estado de \textbf{ERROR} así como las transiciones faltantes. Otra cuestion que me causo un poco de conflicto fue entender completamente la transformación de una expresión regular a un AFN-$\epsilon$, sin duda es algo que se tiene que intentar varias veces para lograr comprender al menos un 99$\%$. Otro problema que encontré fue que no quise usar fotografías en el desarrollo de la practica, es decir, quise trabajar de una forma mas limpia y utilice la librería de \LaTeX \textit{tikz}[4] para poder generar mis autómatas en código \LaTeX. Como consecuencia todos estos problemas llevaron a que me tomara mas tiempo del que esperaba el realizar este reporte y el algoritmo que diera resolución a la practica.

\section{Conclusiones}

Es muy interesante ver un programa que genere autómatas y valide cadenas dado una quintupla en acción. En esta ocasión haciendo uso de un algoritmo[1] se pudo cumplir con el objetivo.

\pagebreak