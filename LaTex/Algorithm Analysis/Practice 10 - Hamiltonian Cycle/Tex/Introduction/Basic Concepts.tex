\section{Basic Concepts:}

{\bfseries Hamiltonian Cycle Definition:} If {\itshape G = ( V, E )} is a graph or multigraph with $\vert$ V $\vert$ $\geq$ 3, we say that {\itshape G} has a {\itshape Hamilton Cycle} if there is a cycle in {\itshape G} that contains every vertex in {\itshape V}. \hfill \break

{\bfseries Theorem:} Let {\itshape G = ( V, E )} be a loop-free undirected graph with $\vert$ V $\vert$ = n $\geq$ 3. If {\itshape deg ( x ) + deg ( y ) $\geq$ n} for all nonadjacent {\itshape x, y $\in$ V}, then {\itshape G} contains a {\itshape Hamilton Cycle}. \hfill \break

{\bfseries Corollary:} If {\itshape G = ( V, E )} be a loop-free undirected graph with $\vert$ V $\vert$ = n $\geq$ 3 and if {\itshape deg ( v ) $\geq$ $\frac{n}{2}$} for all {\itshape v $\in$ V}, then {\itshape G} has a {\itshape Hamilton Cycle}. \hfill \break

{\bfseries Corollary:} If {\itshape G = ( V, E )} be a loop-free undirected graph with $\vert$ V $\vert$ = n $\geq$ 3 and if {\itshape $\vert$ E $\vert$ $\geq$ ${n - 1 \choose 2}$ + 2}, then {\itshape G} has a {\itshape Hamilton Cycle}.

\pagebreak