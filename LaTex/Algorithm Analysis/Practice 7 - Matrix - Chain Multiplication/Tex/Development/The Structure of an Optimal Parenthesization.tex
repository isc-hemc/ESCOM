\subsection{Step 1: The Structure of an Optimal Parenthesization.}

For our first step in the dynamic-programming paradigm, we find the optimal sub-structure and then use it to construct an optimal solution to the problem from optimal solutions to sub-problems. In the matrix-chain multiplication problem, we can perform this step as follows. For convenience, let us adopt the  $A_{i ... j}$, where i $\leq$ j, for the matrix that results from evaluating the product $A_{i} A_{i+1} ... A_{j}$. Observe that if the problem is nontrivial, i.e., i $<$ j, then to parenthesize the product $A_{i} A_{i+1} ... \cdot A_{j}$, we must split the product between $A_{k}$ and $A_{k+1}$ for some k in the range i $\leq$ k $<$ j. That is, for some value of k, we first compute the matrices $A_{i...k}$ and $A_{k+1...j}$ and then multiply them together to produce the final product $A_{i...j}$. The cost of parenthesizing this way is the cost of computing the matrix $A_{i...k}$, plus the cost of computing $A_{k+1...j}$, plus the cost of multiplying them together. \hfill \break

The optimal substructure of this problem is as follows. Suppose that to optimally parenthesize $A_{i} A_{i+1} ... A_{j}$, we split the product between $A_{k}$ and $A_{k+1}$. Then the way we parenthesize the “prefix” sub-chain $A_{i} A_{i+1} ... A_{k}$ within this optimal parenthesization of $A_{i} A_{i+1} ... A_{j}$ must be an optimal parenthesization of $A_{i} A_{i+1} ... A_{k}$. Why? If there were a less costly way to parenthesize $A_{i} A_{i+1} ... A_{k}$ then we could substitute that parenthesization in the optimal parenthesization $A_{i} A_{i+1} ... A_{j}$ to produce another way to parenthesize $A_{i} A_{i+1} ... A_{j}$ whose cost was lower than the optimum: a contradiction. A similar observation holds for how we parenthesize the sub-chain $A_{k+1} A_{k+2} ... A_{j}$ in the optimal parenthesization of $A_{i} A_{i+1} ... A_{j}$: it must be an optimal parenthesization of $A_{k+1} A_{k+2} ... A_{j}$. \hfill \break

Now we use our optimal substructure to show that we can construct an optimal solution to the problem from optimal solutions to sub-problems. We have seen that any solution to a nontrivial instance of the {\bfseries matrix-chain multiplication} problem requires us to split the product, and that any optimal solution contains within it optimal solutions to sub-problem instances. Thus, we can build an optimal solution to an instance of the matrix-chain multiplication problem by splitting the problem into two sub-problems (optimally parenthesizing $A_{i} A_{i+1} ... A_{k}$ and $A_{k+1} A_{k+2} ... A_{j}$), finding optimal solutions to sub-problem instances, and then combining these optimal sub-problem solutions. We must ensure that when we search for the correct place to split the product, we have considered all possible places, so that we are sure of having examined the optimal one.

\pagebreak