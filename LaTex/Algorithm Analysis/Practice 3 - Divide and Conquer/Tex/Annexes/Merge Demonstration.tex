\subsection{Merge Demonstration:}

Demonstrate that Merge algorithm has {\bfseries\itshape linear} order:  \hfill \break

{{\bfseries\color{Violet}{function}} MERGE ( A, p, q, r ):
\begin{lstlisting}
	n1 = q - p + 1
	n2 = r - q
	Let L [ 1, ..., n1 + 1 ] and R [ 1, ..., n2 + 1 ] be new arrays.
	for i = 1 to n1 do
		L [ i ] = A [ p + i - 1 ]
	for j = 1 to n2 do 
		R [ j ] = A [ q + j ]
	i = 0
	j = 0
	for k = p to r do
		if L [ i ] <= R [ j ]
			A [ k ] = L [ i ]
			i++
		else A [ k ] = R [ j ]
			j++
\end{lstlisting} \hfill

\begin{itemize}
\item {\bfseries\itshape\color{Maroon}{Demonstration:}} 
\end{itemize} 

\begin{multicols}{2}
\begin{itemize}
\item {\bfseries\itshape\color{Violet}{Analyzing the complexity of each line:}}
\begin{enumerate}
\item Line 1, 2, 3 = $\theta\ (\ 1\ )$.
\item Line 4 = $\theta\ (\ n_{1}\ )$.
\begin{tasks}
\task Line 5 = $\theta\ (\ 1\ )$.
\end{tasks}
\item Line 6 = $\theta\ (\ n_{2}\ )$.  
\begin{tasks}
\task Line 7 = $\theta\ (\ 1\ )$.
\end{tasks}
\item Line 8, 9 = $\theta\ (\ 1\ )$. {\Large\hspace{3.5cm}$\Longrightarrow$}
\item Line 10 = $\theta\ (\ r\ -\ p\ +\ 1\ )$ = $\theta\ (\ n_{3}\ )$.
\begin{tasks}
\task Line 11 = $\theta\ (\ 1\ )$.
\task Line 12 = $\theta\ (\ 1\ )$.
\task Line 13 = $\theta\ (\ 1\ )$.
\task Line 14 = $\theta\ (\ 1\ )$.
\task Line 15 = $\theta\ (\ 1\ )$.
\end{tasks}
\end{enumerate}
\hfill \break \break \break \break \break

\begin{enumerate}
\item Line 1, 2, 3 = $\theta\ (\ 1\ )$.
\item Line 4 - 5 = $\theta\ (\ n_{1}\ *\ 1\ )$ = $\theta\ (\ n_{1}\  )$.
\item Line 6 - 7 = $\theta\ (\ n_{2}\ *\ 1\ )$ = $\theta\ (\ n_{2}\  )$.
\item Line 8, 9 = $\theta\ (\ 1\ )$. 
\item Line 10 - 11 - 12 - 13 - 14 - 15 = $\theta\ (\ n\ *\ 1\ *\ 1\ *\ 1\ *\ 1\ *\ 1\ )$ = $\theta\ (\ n_{3}\  )$.
\end{enumerate}
\hfill \break \break \break \break \break  
\end{itemize} 
\end{multicols} \hfill

\begin{itemize}
\item {\bfseries\itshape\color{Violet}{Then, from all lines:}}
\end{itemize} \hfill

\begin{ceqn}
\begin{align}
T\ (\ n\ )\ =\ \theta\ (\ n_{1}\ +\ n_{2}\ +\ n_{3}\ +\ 1\ +\ 1\ )\ =\ \theta\ (\ n\ )
\end{align}
\end{ceqn} \hfill

\begin{itemize}
\item {\bfseries\itshape\color{Violet}{Finally:}}
\end{itemize} \hfill

\begin{ceqn}
\begin{align}
Merge\ \in\ \theta\ (\ n\ )
\end{align}
\end{ceqn}
\pagebreak