\subsection{Maximum Crossing Subarray:}

Demonstration that maximum crossing subarray has linear complexity: {\bfseries\itshape T ( n ) = ( n ) } order. \hfill \break

{\bfseries\color{Violet}{function}} FIND-MAX-CROSSING-SUBARRAY (A, low, mid, high)
\begin{lstlisting}[mathescape=true]
left_sum = -$\infty$
sum = 0
for i = mid downto low
	sum = sum + A[i]
	if sum > left_sum
		left_sum = sum
		max_left = i
rigth_sum = -$\infty$
sum = 0
for j = mid + 1 to high 
	sum = sum + A[j]
	if sum > right_sum
		right_sum = sum
		max_right = j
return ( max_left, max_right, left_sum + right_sum )
\end{lstlisting} \hfill

\begin{itemize}
\item {\bfseries\itshape\color{carmine}{Demonstration:}} 
\end{itemize} 

\begin{itemize}
\item {\bfseries\itshape\color{Violet}{Analyzing the complexity of each line:}}
\begin{enumerate}
\item Line 1, 2 = $\theta\ (\ 1\ )$.
\item Line 3, 4, 5, 6, 7 = $\theta\ (\ \frac{n}{2}\ )$.
\item Line 8, 9 = $\theta\ (\ 1\ )$.
\item Line 10, 11, 12, 13, 14 = $\theta\ (\ \frac{n}{2}\ )$.
\item Line 15 = $\theta\ (\ 1\ )$.
\end{enumerate}
\end{itemize} \hfill

\begin{itemize}
\item {\bfseries\itshape\color{Violet}{Then, from all lines:}}
\end{itemize} \hfill

\begin{ceqn}
\begin{align}
T\ (\ n\ )\ =\ \theta\ (\ \frac{n}{2}\ +\ \frac{n}{2}\ )\ =\ \theta\ (\ n\ )
\end{align}
\end{ceqn} \hfill

\begin{itemize}
\item {\bfseries\itshape\color{Violet}{Finally:}}
\end{itemize} \hfill

\begin{ceqn}
\begin{align}
Maximum\ Crossing\ Subarray\ \in\ \theta\ (\ n\ )
\end{align}
\end{ceqn}
\pagebreak

