\subsection{Subtracting Amplifier:}

The circuit that we are going to analyze it's the one in Figure 3.5.0 using the following parameters: \hfill \break

{\bfseries
\begin{itemize}
\item $R_{f}$ = 150 K $\Omega$
\item $R_{1}$ = 100 K $\Omega$
\item $R_{2}$ = 100 K $\Omega$
\item $R_{div\ 1}$ = 1 K $\Omega$
\item $R_{div\ 2}$ = 1 K $\Omega$
\item $R_{div\ 3}$ = 2.2 K $\Omega$
\end{itemize}} \hfill

{\bfseries\itshape\color{Violet}{
\begin{itemize}
\item For $E_{2}$ using a voltage divider:
\end{itemize}}} 

\begin{flushright}
{\bfseries\itshape\color{carmine}{Formula: $E_{n}\ =\ (\ \frac{(	\ R_{n\ )(\ E\ )}}{R_{1}\ +\ R_{2}\ +\ ...\ +\ R_{n}}\ )$:}} \hfill \break
\end{flushright}

\begin{ceqn}
\begin{align*}
E_{2}\ &=\ \frac{(\ R_{div\ 1}\ ) \cdot (\ 12\ V\ )}{R_{div\ 1}\ +\ R_{div\ 2}\ +\ R_{div\ 3}} \\ \\
&=\ \frac{(\ 1\ K\ \Omega\ ) \cdot (\ 12\ V\ )}{1\ K\ \Omega\ +\ 1\ K\ \Omega\ +\ 2.2\ K\ \Omega} \\ \\
&=\ 2.8\ V
\end{align*}
\end{ceqn} \hfill \break

{\bfseries\itshape\color{Violet}{
\begin{itemize}
\item For $E_{1}$ using a voltage divider:
\end{itemize}}} 

\begin{ceqn}
\begin{align*}
E_{1}\ &=\ E_{2}\ +\ \frac{(\ R_{div\ 2}\ ) \cdot (\ 12\ V\ )}{R_{div\ 1}\ +\ R_{div\ 2}\ +\ R_{div\ 3}} \\ \\
&=\ 2.8\ V\ +\ \frac{(\ 1\ K\ \Omega\ ) \cdot (\ 12\ V\ )}{560 \Omega\ +\ 1\ K\ \Omega\ +\ 2.2\ K\ \Omega} \\ \\
&=\ 2.8\ V\ +\ \frac{(\ 1\ K\ \Omega\ ) \cdot (\ 12\ V\ )}{1\ K \Omega\ +\ 1\ K\ \Omega\ +\ 2.2\ K\ \Omega} \\ \\
&=\ 5.65\ V
\end{align*}
\end{ceqn} \hfill \break

{\bfseries\itshape\color{Violet}{
\begin{itemize}
\item Then, for $V_{sal}$:
\end{itemize}}} 

\begin{flushright}
{\bfseries\itshape\color{carmine}{Formula: $V_{sal}\ =\ R_{f}\ (\ \frac{E_{1}}{R_{1}}\ -\ \frac{E_{2}}{R_{2}}\ -\ ...\ -\ \frac{E_{n}}{R_{n}}\ )$:}} \hfill \break
\end{flushright}

\begin{ceqn}
\begin{align*}
V_{sal}\ &=\ 150\ K\ \Omega\ (\ \frac{5.65\ V}{100\ K\ \Omega}\ -\ \frac{2.8\ V}{100\ K\ \Omega}\ ) \\ \\
&=\ 4.27\ V
\end{align*}
\end{ceqn} \hfill \break

\pagebreak