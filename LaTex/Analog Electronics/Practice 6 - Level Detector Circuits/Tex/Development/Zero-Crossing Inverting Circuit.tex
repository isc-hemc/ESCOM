\subsection{Inverting Zero-Crossing Level Detector:}

Setting the {\itshape waveform generator} in a sinusoidal signal with a frequency of 1 KHz and 16 $V_{pp}$ we connect the positive terminal of the {\itshape generator} to the $V_{i}$ terminal of the circuit in Figure 3.2.0 and the negative terminal to the common ground. Then, once the respectively sources in the terminals 7 and 4 were connected, we turned on the {\itshape generator} and the voltage sources, thus, connecting the channel 1 of the oscilloscope in $V_{i}$ and the channel 2 in the $V_{o}$ we registered the waveform in Figure 3.2.1. Finally, we change the oscilloscope mode to X-Y and captured the transfer function in Figure 3.2.2. \hfill \break

{\bfseries\itshape\color{carmine}{Observation:}} {\itshape\color{carmine}{In the practice format, the peak-to-peak voltage for this circuit was of 5 $V_{pp}$, but the first circuit we measure was the one with {\bfseries Hysteresis}, then, we forget to modify the waveform voltage, so we use a 16 $V_{pp}$ voltage.}} \hfill

\begin{figure}[H]
\includegraphics[width = 8cm, height = 6cm]{c2.png}
\centering \linebreak \linebreak Figure 3.2.0: Inverting zero-crossing circuit.
\end{figure} \hfill

\begin{multicols}{2}
\begin{figure}[H]
\includegraphics[width = 8cm, height = 5cm]{o2.png}
\centering \linebreak \linebreak Figure 3.2.1: Input and output waveform.
\end{figure} \hfill

\begin{figure}[H]
\includegraphics[width = 8cm, height = 5cm]{t2.png}
\centering \linebreak \linebreak Figure 3.2.2 Transfer function for Figure 3.2.1.
\end{figure} \hfill
\end{multicols} 

{\bfseries\itshape\color{carmine}{Observation:}} {\itshape\color{carmine}{In Figure 3.2.1 we can see two waveform, the yellow one it's the input of the circuit, analogously, the blue it's for the output.}}

\pagebreak