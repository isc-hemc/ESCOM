\section{Questionnaire:}

\begin{itemize}
\item {\bfseries\itshape Mentions the importance of voltage rectifiers:} We note that rectifiers are very important in all applications, since they can convert an AC signal into direct current, it also has uses like doubling, tripling or quadrupling voltage and these functions are used daily in many devices and electronic tools.

\item {\bfseries\itshape Explains the difference between a half-wave rectifier and a full-wave rectifier:}
Half-wave rectifiers operate by passing half of the alternating current through one or more diodes, converting this half of the alternating current into direct electric current. Half-wave rectifiers are not very efficient because they only convert half of the alternating current into direct current. half-wave rectifiers are much less complicated and require only a diode for their operation.

\item {\bfseries\itshape What is the difference of a full wave rectifier with central bypass and bridge type:} The full-wave rectifier with central bypass uses both halves of the input sine wave; to obtain a unipolar output, reverses the negative semi cycles of the sinusoidal wave. In this application it is used in the central winding of the transformer in order to obtain two equal voltages in parallel with the two halves of the secondary winding. The bridge-type rectifier does not have variations in the output signal with respect to the rectifier with central bypass, the difference is that it does not use bypass but two more diodes. Its operation is that during the positive half-cycles of the input voltage the current is conducted through the diode1, the load and diode 2 (because it is positive). Meanwhile the diodes 3 and 4 are reverse biased.

\item {\bfseries\itshape How to measure the output voltage of the rectifier?} The output voltage of the rectifier can be measured with the oscilloscope by looking at the peak voltage (Vp) passing through the indicated or selected resistance the voltage is reduced according to the diodes that are by the passage of current flow

\item {\bfseries\itshape How to measure the ripple voltage of the rectifier?} Is called the ripple voltage to the difference between the maximum voltage and the minimum voltage of the output waveform of the DC voltage source of a rectifier that uses a capacitor. Basically the ripple voltage it's the peak-to-peak voltage, so, the easier way to measure it, it's with the oscilloscope, in the option {\bfseries\itshape measures} in the section $V_{pp}$, the oscilloscope must be connected in the terminals of $R_{L}$ ( for this case ).  
\end{itemize}
\pagebreak